\chapter{Ausblick}

\section{Verbreitung}

SQLAlchemy wird in einer Vielzahl von Projekten eingesetzt und auch von
namhaften Firmen verwendet.

Die bekanntesten Firmen, die SQLAlchemy einsetzen, sind
\href{http://www.cern.ch}{CERN} und
\href{http://www.mozilla.com}{Mozilla}.

Am CERN wird SQLAlchemy im Rahmen des CMS\footnote{Compact Muon
Solenoid}-Projekts am LHC\footnote{Large Hadron Collider} eingesetzt, um über
ein Webinterface Daten des Experiments darstellen zu können, die außer in einer
Datenbank teilweise nur auf nicht direkt mit dem Internet verbundenen Rechnern
gespeichert sind. \\
Mozilla setzt SQLAlchemy im Rahmen des \emph{Firefox Sync}-Systems ein, mit dem
man seine Bookmarks, History und Passwörter verschlüsselt bei Mozilla speichern
und zwischen verschiedenen PCs synchronisieren kann.

Die \href{http://fedoraproject.org/wiki/Infrastructure/Services}{Fedora
Foundation} setzt SQLAlchemy für einen Großteil der Infrastruktur ein.

Ein ebenfalls in der Linux-Community angesiedeltes Projekt ist
\href{http://screenshots.debian.net/}{screenshots.debian.net}, wo Debian
und Ubuntu-Nutzer Screenshots von ihren Desktops oder Programmen hochladen
können. SQLAlchemy wird dort für alle Zugriffe auf das Datenbank-Backend der
Website benutzt.

Natürlich gibt es noch deutlich mehr Firmen und Projekte, die SQLAlchemy
verwenden. Eine vollständigere Liste, die auch diverse auf SQLAlchemy basierende
Bibliotheken enthält, gibt es im
\href{http://www.sqlalchemy.org/trac/wiki/SAApps}{SQLAlchemy-Wiki}.


\section{Unterschiede zu anderen ORMs}

Das neben SQLAlchemy bekannteste ORM für Python ist
\href{http://www.djangoproject.com/}{Django}. Allerdings handelt es sich bei
Django nicht nur um ein ORM sondern um ein komplettes Webframework. Die
Datenbankstruktur wird im Django-ORM ähnlich wie bei der deklarativen
SQLAlchemy-Syntax modelliert; es sind also keine XML-Dateien o.ä. dazu
notwendig.

Auch \href{http://www.sqlobject.org}{SQLObject}, ein weiteres ORM für Python,
benutzt eine SQLAlchemy/Django-ähnliche Syntax, die es ermöglicht, eine
Datenbank ausschließlich mit Python-Code zu modellieren. Wie auch schon Django
bietet auch SQLObject ausschließlich eine \emph{deklarative} Syntax an und damit
keine Möglichkeit, Tabellen und Mappingobjekte separat zu definieren.

In der \href{http://www.php.net}{PHP}-Welt ist
\href{http://www.propelorm.org}{Propel} ein verbreitetes ORM. Anders als bei den
zuvor beschriebenen ORMs ist es bei Propel notwendig, die Datenbankstruktur in
einer XML-Datei zu definieren. Anhand dieser XML-Datei kann danach mit dem
Script \texttt{propel-gen} der PHP-Code für die Objektmodelle generiert werden.
Ein weiteres Feature von \texttt{propel-gen} ist das Generieren einer SQL-Datei
die zum Erstellen der in der XML-Datei definierten Tabellen genutzt werden kann
- alternativ kann man den erzeugten SQL-Code auch direkt ausführen lassen statt
ihn in eine Datei zu schreiben.

Das wohl bekannteste ORM ist \href{http://www.hibernate.org/}{Hibernate} für
Java bzw. \href{http://nhforge.org/}{NHibernate} für das .NET-Framework. Auch
dort werden die Tabellen in XML-Dateien definiert und danach mit einem externen
Programm Klassen und/oder die Datenbanktabellen erzeugt. Ähnlich wie SQLAlchemy
bietet auch Hibernate mit der HQL\footnote{Hibernate Query Language} eine
Möglichkeit, SQL auf einer niedrigen Ebene zu generieren ohne SQL-Abfragen in
Stringform schreiben zu müssen.


\section{Fazit}

SQLAlchemy bietet sowohl Entwicklern, die noch nicht viel mit Datenbanken zu tun
hatten, als auch erfahrenen Datenbankentwicklern, die SQL als zweite
Muttersprache bezeichnen können eine komfortable objektrelationale
Datenbankschnittstelle.

Durch die Tatsache, dass zur Datenbankdefinition keine XML-Dateien mit einer
erst zu lernenden Struktur benötigt werden sondern ganz normaler Python-Code
geschrieben werden kann, hat man sich auch schnell in SQLAlchemy eingearbeitet.

Wenn man ein bereits existierendes Projekt auf SQLAlchemy umstellen will ist
auch dies ohne Weiteres möglich, da man keinen generierten Code, der sich evtl.
über viele einzelne Dateien erstreckt, hat sondern nur eine einzelne Datei mit
den Datenbankdefinitionen (oder auch mehrere Dateien sofern man die Definitionen
einer großen Datenbank auf mehrere Dateien verteilen möchte) hinzufügen und
einbinden muss und dann sofort eine vollständige ORM-Unterstützung hat. Dank der
Möglichkeit, komplett handgeschriebene SQL-Statements abzuschicken kann man auch
ohne Weiteres die bereits vorhandenen Datenbankabfragen über SQLAlchemy laufen
lassen und sie schrittweise auf ORM umstellen.

Kurz gesagt, SQLAlchemy ist ein komfortables ORM-System und wenn man mit Python
arbeitet und nicht gerade schon durch Django ein anderes ORM benutzt ist es auf
jeden Fall einen Blick wert - ob man letzendlich ORM nutzen will oder nicht muss
man dann für sich selbst entscheiden. Allerdings ist die Zeitersparnis
insbesondere bei CRUD\footnote{\textbf{C}reate \textbf{R}ead \textbf{U}pdate
\textbf{D}elete}-Standardaufgaben enorm.
