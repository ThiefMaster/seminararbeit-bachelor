\chapter{Ausblick}

\section{Verbreitung}

SQLAlchemy wird in einer Vielzahl von Projekten eingesetzt und auch von
namhaften Firmen verwendet.

Die bekanntesten Firmen, die SQLAlchemy einsetzen, sind
\href{http://www.cern.ch}{CERN} und
\href{http://www.mozilla.com}{Mozilla}.

Am CERN wird SQLAlchemy im Rahmen des CMS\footnote{Compact Muon
Solenoid}-Projekts am LHC\footnote{Large Hadron Collider} eingesetzt, um über
ein Webinterface Daten des Experiments darstellen zu können, die außer in einer
Datenbank teilweise nur auf nicht direkt mit dem Internet verbundenen Rechnern
gespeichert sind. \\
Mozilla setzt SQLAlchemy im Rahmen des \emph{Firefox Sync}-Systems ein, mit dem
man seine Bookmarks, History und Passwörter verschlüsselt bei Mozilla speichern
und zwischen verschiedenen PCs synchronisieren kann.

Die \href{http://fedoraproject.org/wiki/Infrastructure/Services}{Fedora
Foundation} setzt SQLAlchemy für einen Großteil der Infrastruktur ein.

Ein ebenfalls in der Linux-Community angesiedeltes Projekt ist
\href{http://screenshots.debian.net/}{screenshots.debian.net}, wo Debian
und Ubuntu-Nutzer Screenshots von ihren Desktops oder Programmen hochladen
können. SQLAlchemy wird dort für alle Zugriffe auf das Datenbank-Backend der
Website benutzt.

Natürlich gibt es noch deutlich mehr Firmen und Projekte, die SQLAlchemy
verwenden. Eine vollständigere Liste, die auch diverse auf SQLAlchemy basierende
Bibliotheken enthält, gibt es im
\href{http://www.sqlalchemy.org/trac/wiki/SAApps}{SQLAlchemy-Wiki}.

\section{Unterschiede zu anderen ORMs}

\section{Fazit}

