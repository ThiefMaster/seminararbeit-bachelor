\chapter{SQLAlchemy - praktische Nutzung}

Die sich derzeit noch in Entwicklung befindliche neue Version der Website des
\href{http://www.gamesurge.net}{GameSurge IRC-Netzes} verwendet eine angepasste
Version des \href{https://github.com/gmr/tinman}{Tinman-Frameworks}, welches auf
dem \href{http://www.tornadoweb.org/}{Tornado-Framework/Webserver} von Facebook
basiert. Die wichtigste Anpassung bei Tinman betrifft den SQLAlchemy-Code des
Datenbankmoduls, der standardmäßig immer \texttt{autocommit} benutzt, was, wie in
\autoref{sqlalchemy-orm} erwähnt, problematisch sein kann. In der angepassten
Version kann \texttt{autocommit} per Configdatei deaktiviert werden, was auch
notwendig ist, damit die Website korrekt funktioniert.

Das erste Beispiel - die dazugehörige Tabelle wird in
\autoref{lst:gs-model-trustrequest} definiert - ist aus einem Interface, über
das Benutzer gewisse Privilegien, sog. \emph{Trusts}, für ihre IP-Adresse
beantragen können. Da die Website PostgreSQL als Datenbankbackend nutzt und sich
dies mit größter Wahrscheinlichkeit nicht ändern wird, wird die IP-Adresse in
einem Datenbankfeld des Typs \texttt{postgresql.INET} gespeichert. Am Namen des
Datentypen sieht man bereits, dass es sich hierbei um einen
datenbankspezifischen Typen handelt, der nicht mit anderen Datenbanksystemen
funktionieren wird. Allerdings hat er den Vorteil, dass er sowohl IPv4- als auch
IPv6\footnote{\emph{Internet Protocol Version 6}, dringend benötigter Nachfolger
des aktuellen IPv4 mit einem weitaus größeren Adressraum von $2^{128}$ statt
$2^{32}$}-Adressen effizient speichern kann und IP-spezifische Operationen wie
eine Suche nach IPs innerhalb eines gewissen Subnetzes unterstützt.
Der Parameter \texttt{server\_default} der Spalte \texttt{request\_time} gibt
an, dass beim Erstellen der Tabelle ein serverseitiger Defaultwert angegeben
wird, der durch die SQL-Funktion \texttt{current\_timestamp()} erzeugt wird.
Das zum Erstellen der Tabelle generierte SQL enthält also \texttt{DEFAULT
current\_timestamp()}. Alternativ zum \texttt{server\_default}-Parameter gäbe es
auch \texttt{default}. Diese Option würde der Spalte in der Datenbank keinen
Defaultwert zuweisen, sondern ihn explizit im \texttt{INSERT}-Statement setzen.

\lstinputlisting[language=Python,label=lst:gs-model-trustrequest,
caption=TrustRequest-Tabellendefinition]{code/gs-model-trustrequest.py}

\autoref{lst:gs-trust-request} ist der Code hinter dem für Benutzer zugänglichen
Formular. Um zu verhindern, dass ein \emph{Trust} erneut beantragt wird, solange
der erste Antrag noch nicht bearbeitet wurde, wird die Anzahl der offenen
Anträge für die angegebene IP aus der Datenbank ausgelesen und sofern sie nicht
\texttt{0} ist mit einer Fehlermeldung abgebrochen. Danach wird das neue
\texttt{TrustRequest}-Objekt erstellt und mit Daten gefüllt. Da zusätzlich noch
Daten zu der IP abgefragt werden sollen, die u.U. längere Netzwerkanfragen
benötigen, wird das Objekt noch nicht persistiert sondern zuerst an asynchrone
Funktionen übergeben, die diese Daten abfragen. Nachdem alle Daten abgefragt
und dem Objekt hinzugefügt wurden, wird das Objekt in der Methode
\texttt{post\_resolved()} der Session hinzugefügt und die Datenbanktransaktion
committed.

\lstinputlisting[language=Python,label=lst:gs-trust-request,
caption=Erstellen eines neuen TrustRequest-Objekts]{code/gs-trust-request.py}

Beim Bearbeiten der Anträge ist es oftmals von Interesse, ob der Benutzer
schon andere \emph{Trusts} beantragt hat und ob für die angegebene IP-Adresse
bereits \emph{Trusts} beantragt wurden. Diese Informationen werden vom Code in
\autoref{lst:gs-trust-list} abgerufen. In der dort verwendeten Query sieht man
gut die Vermischung von ORM und klassischen SQL-Konstrukten für komplexe
Abfragen: Die \texttt{TrustRequest}-Objekte selbst werden über die normalen
ORM-Funktinen aus der Datenbank abgerufen. Allerdings wird die Query um
\texttt{ip\_reqs.as\_scalar()} und \texttt{user\_reqs.as\_scalar()} erweitert.
Die Methode \texttt{as\_scalar()} gibt ein \texttt{Query}-Objekt in einer
Subquery-kompatiblen Form zurück. Ebenfalls erwähnenswert ist das Erstellen der
Subqueries. Da es notwendig ist, in der \texttt{WHERE}-Bedingung der Subqueries
eine Spalte aus derselben Tabelle der Haupt-Query zu referenzieren, muss die
Tabelle der Subquery über einen Alias referenziert werden. Dies geschieht mit
der \texttt{aliased()}-Funktion.
Beim Iterieren über das Resultset bekommt man für die einzelnen Zeilen nun nicht
mehr einfach das Mapping-Objekt sondern ein Tupel, welches alle im
\texttt{query()}-Aufruf angegebenen Elemente enthält, d.h. das Mapping-Objekt
für den \texttt{TrustRequest} und von \texttt{ip\_reqs} und \texttt{user\_reqs}
jeweils einen Integer.

\lstinputlisting[language=Python,label=lst:gs-trust-list,
caption=Abrufen von TrustRequest-Objekten und zus\a"tzlichen
Informationen]{code/gs-trust-list.py}

Die Website soll ebenfalls die Möglichkeit bieten, News zu posten und
nachträglich in andere Sprachen zu übersetzen. Dabei sollen die
englischsprachigen News und die Übersetzungen in verschiedenen Tabellen
gespeichert werden. Um anhand einem \texttt{News}-Objekts leicht an eine
bestimmte Übersetzung zu kommen ist es wünschenswert, bei der Relation von
\texttt{News} nach \texttt{NewsTranslation} keine Liste mit numerischen Indizes
sondern ein \texttt{dict} mit dem Sprachkürzel als Index zu haben. Dies wird in
\autoref{lst:gs-model-news} mithilfe des \texttt{collection\_class}-Parameters
von \texttt{relationship()} realisiert. \\
\texttt{attribute\_mapped\_collection('language')} gibt an, dass das Feld
\texttt{language} als Index benutzt werden soll. \\
Zusätzlich wurde mit
\texttt{cascade='all, delete, delete-orphan'} festgelegt, dass "verwaiste"
Übersetzungen, d.h. Übersetzungen, deren Original-News gelöscht wurde, ebenfalls
gelöscht werden. Der Fremdschlüssel übernimmt dies dank
\texttt{ondelete='CASCADE'} zwar bereits, allerdings würde SQLAlchemy ohne
die Kaskadierung versuchen, die \texttt{newsid} der \texttt{NewsTranslation} auf
\texttt{NULL} zu setzen, was an \texttt{nullable=False} scheitern würde und auch
nicht gewollt ist.

\lstinputlisting[language=Python,label=lst:gs-model-news,
caption=News-Tabellendefinition]{code/gs-model-news.py}

Im News-Template (\autoref{lst:gs-news-tpl}) wird bei der Ausgabe dann einfach
geprüft, ob es für die aktuelle Sprache eine Übersetzung gibt und danach
entweder die Übersetzung oder das englische Original ausgegeben.

\lstinputlisting[language=Python,label=lst:gs-news-tpl,caption=News-Template]
{code/gs-news-tpl.html}
