\documentclass[fontsize=12pt,a4paper,headinclude,oneside,halfparskip,pdftex,pointlessnumbers,plainheadsepline]{scrreprt} % headinclude
\usepackage[automark]{scrpage2}
\typearea[2mm]{15}
% Schriften
%\usepackage{lmodern}
\usepackage{mathpazo} % Palatino für Mathemodus
\usepackage{courier}
\usepackage{amsmath,marvosym} % Mathesachen
\usepackage{setspace} % 1,5 Zeilenabstand
\onehalfspacing
\usepackage[T1]{fontenc} % Ligaturen, richtige Umlaute im PDF
\usepackage[utf8]{inputenc} % latin1-Kodierung für Umlaute
\usepackage{ngerman} % neue deutsche Rechtschreibung
\usepackage[ngerman]{babel} % Silbentrennung
\clubpenalty = 10000 % Schusterjungen vermeiden
\widowpenalty = 10000 % Hurenkinder vermeiden
\displaywidowpenalty = 10000 % und nochmal für Formeln
% PDF
\usepackage[plainpages=false,ngerman,pdfpagelabels=true,bookmarks=true,
bookmarksopen=true,colorlinks=true,linkcolor=blue,pdfauthor={Adrian Mönnich},
pdftitle={ORM mit SQLAlchemy}]{hyperref}
% \usepackage{pdflscape} % einzelne Seiten drehen können
\usepackage{microtype} % spezielle PDF-mikro-typographische Optimierungen
\newcommand{\subfigureautorefname}{\figurename} % um \autoref auch für subfigures benutzen
\usepackage{url}
% Tabellen
\usepackage{multirow} % Tabellen-Zellen über mehrere Zeilen
\usepackage{multicol} % mehre Spalten auf eine Seite
\usepackage{tabularx} % Für Tabellen mit vorgegeben Größen
\usepackage{array}
\usepackage{float}
%  Biblio
%\usepackage{natbib}%\usepackage[round]{natbib}
%\usepackage{bibgerm} % Umlaute in BibTeX
% Bilder
\usepackage{graphicx} % Bilder
\usepackage{color} % Farben
%\usepackage[sf,SF]{subfigure} % mehrere Abbildungen nebeneinander/übereinander
\usepackage{subfigure} % mehrere Abbildungen nebeneinander/übereinander
%\usepackage[width=0.8\textwidth,font=small,labelfont={sf,bf},textfont=sf]{caption} % Beschreibungen
\usepackage[width=0.8\textwidth,font=small,labelfont={bf}]{caption} % Beschreibungen
%\usepackage{ifthen} % für Bedingen, hier bei \newcommands benötigt

% Quellcode
\usepackage{listings} % für Formatierung in Quelltexten
%\usepackage{minted}
%\usemintedstyle{default}
%\renewcommand{\theFancyVerbLine}{\sffamily\textcolor[rgb]{0.5,0.5,0.5}{\scriptsize\arabic{FancyVerbLine}}}
%\newmintedfile{python}{linenos,numbersep=5pt,frame=lines,framesep=2mm}
%\newminted{python}{linenos,numbersep=5pt,frame=lines,framesep=2mm}
%\def\listingautorefname{Listing}

\definecolor{codebg}{rgb}{0.98,0.98,0.98}
\lstset{
  language=Python,                % choose the language of the code
  basicstyle=\small,       % the size of the fonts that are used for the code
  numbers=left,                   % where to put the line-numbers
  numberstyle=\small,      % the size of the fonts that are used for the line-numbers
  numbersep=5pt,                  % how far the line-numbers are from the code
  backgroundcolor=\color{codebg} ,% choose the background color. You must add \usepackage{color}
  showspaces=false,               % show spaces adding particular underscores
  showstringspaces=false,         % underline spaces within strings
  showtabs=false,                 % show tabs within strings adding particular underscores
  frame=lines,                    % adds a frame around the code
  framesep=2mm,
  tabsize=4,                      % sets default tabsize to 2 spaces
  captionpos=b,                   % sets the caption-position to bottom
  breaklines=true,                % sets automatic line breaking
  breakatwhitespace=true,         % sets if automatic breaks should only happen at whitespace
  numberbychapter=false,
  aboveskip=0.75cm,
  belowskip=0.5cm,
  inputencoding=latin1
}
\lstset{literate=
  {Ö}{{\"O}}1
  {Ä}{{\"A}}1
  {Ü}{{\"U}}1
  {ß}{{\ss}}2
  {ü}{{\"u}}1
  {ä}{{\"a}}1
  {ö}{{\"o}}1
}

% für autoref von Gleichungen in itemitemize-Umgebeungen
\makeatletter
\newcommand{\saved@equation}{}
\let\saved@equation\equation
\def\equation{\@hyper@itemfalse\saved@equation}
\makeatother


%\setcaphanging
% linksbündige Fußboten
\deffootnote{1.5em}{1em}{\makebox[1.5em][l]{\thefootnotemark}}

% Standardschrift für Überschriften
\setkomafont{sectioning}{\normalfont\normalcolor\bfseries}
\setkomafont{pagehead}{\normalfont\normalcolor\small}
\setkomafont{descriptionlabel}{\normalfont\normalcolor\bfseries}
%\setkomafont{pagehead}{\normalfont\normalcolor\small\sffamily}
%\setkomafont{section}{\normalfont\normalcolor\Large\bfseries}
%\setkomafont{subsection}{\normalfont\normalcolor\large\bfseries}
%\addtokomafont{caption}{\normalfont\normalcolor\small\sffamily}


%\setcounter{tocdepth}{2}

\usepackage{chngcntr}
\counterwithout{figure}{chapter}


% Eigene Befehle %%%%%%%%%%%%%%%%%%%%%%%%%%%%%%%%%%%%%%%%%%%%%%%%%5
\newcommand{\todotext}[1]{
      {\color{red} TODO: #1} \normalfont
}
% Hinweis auf Programme in Datei
\newcommand{\datei}[1]{
      {\ttfamily{#1}}
}
\newcommand{\code}[1]{
      {\ttfamily{#1}}
}
% bild mit definierter Breite einfügen
\newcommand{\bild}[4]{
  \begin{figure}[!hbt]
    %\begin{center}
    \centering
      \vspace{1ex}
      \includegraphics[width=#2]{images/#1}
      \caption[#4]{\label{img.#1} #3}
    %\end{center}
    \vspace{1ex}
  \end{figure}
}
% bild mit eigener Breite
\newcommand{\bilda}[3]{
  \begin{figure}[!hbt]
    %\begin{center}
    \centering
      \vspace{1ex}
      \includegraphics{images/#1}
      \caption[#3]{\label{img.#1} #2}
      \vspace{1ex}
    %\end{center}
  \end{figure}
}

% Trennsachen
% \hyphenation{Tra-jektorien}


% hier beginnt der eigentliche Inhalt %%%%%%%%%%%%%%%%%%%%%%%%%%%%%%%
\begin{document}
%\pagenumbering{Roman} % große Römische Seitenummerierung
\pagestyle{empty}

% Titelseite %%%%%%%%%%%%%%%%%%%%%%%%%%%%%%%%%%%%%%%%%%%%%%%
\clearscrheadings\clearscrplain


\begin{center}
\begin{Huge}
Hochschule Karlsruhe\\
\vspace{3mm}
\end{Huge}{\Large Fakultät für Informatik}\\

\vspace{20mm}
\begin{Large}
ORM mit SQLAlchemy\\
\end{Large}
\vspace{8mm}
Seminararbeit\\
Wintersemester 2010/11\\
\vspace{0.4cm}
\vspace{2 cm}
Adrian Mönnich \\
Matrikel-Nummer 29189\\
\vspace{8cm}
\begin{tabular}{ll}
{\bf Betreuer} & Prof. Klaus Gremminger\\
\end{tabular}

\end{center}
\clearpage





%% Kopfzeile %%%%%%%%%%%%%%%%%%%%%%%%%%%%%%%
\pagestyle{scrheadings}
\clearscrheadings\clearscrplain
\setheadwidth{text}
\setheadsepline{.4pt}
\automark{chapter}
\ihead[\headmark]{\headmark}
\ohead[\pagemark]{\pagemark}
%%%%%%%%%%%%%%%%%%%%%%%%%%%%%%%%%%%%%%%%%%

\addtocontents{toc}{\protect\thispagestyle{empty}}
\tableofcontents
%\listoffigures \addcontentsline{toc}{chapter}{Abbildungsverzeichnis}
%\listoftables \addcontentsline{toc}{chapter}{Tabellenverzeichnis}

%\chapter*{Symbolverzeichnis}\label{s.sym}
%\addcontentsline{toc}{chapter}{Symbolverzeichnis}
%\markboth{Symbolverzeichnis}{Symbolverzeichnis}
%\section*{Allgemeine Symbole}\label{s.sym.alg}
%\begin{flushleft}\begin{tabularx}{\textwidth}{l|X}
%Symbol & Bedeutung\\\hline
%$a$ & der Skalar $a$ \\
%$\vec{x}$ & der Vektor $\vec{x}$\\
%$\mat{A}$ & die Matrix $\mat{A}$\\
%\end{tabularx}\end{flushleft}

\newpage
\setcounter{page}{1}

%%%%%%%%%%%%%%%%%%%%%%%%%%%%%%%%%%%%%%%%%%%%%%%%%%%%%%%%%%%%%%%%%%
%%%%%%%%%%%%%%%%%%%%%%%%%%%%%%%%%%%%%%%%%%%%%%%%%%%%%%%%%%%%%%%%%%
%%%%%%%%%%%%%%%%%%%%%%%%%%%%%%%%%%%%%%%%%%%%%%%%%%%%%%%%%%%%%%%%%%
%%%%%%%%%%%%%%%%%%%%%%%%%%%%%%%%%%%%%%%%%%%%%%%%%%%%%%%%%%%%%%%%%%
\chapter{Grundlagen}

\section{Was ist ORM?}

Unter objektrelationaler Abbildung
(engl. \textbf{o}bject-\textbf{r}elational \textbf{m}apping)
versteht man eine Technik, mit der man in einer objektorientierten
Programmiersprache Objekte in einer relationalen Datenbank ablegen kann und
später wieder darauf zugreifen kann. Dazu ist es normalerweise nicht
notwendig, Details über die tatsächliche Datenbankstruktur zu kennen oder gar
selbst SQL\footnote{\emph{Structured Query Language}, eine standardisierte
Datenbanksprache}-Abfragen zu schreiben.

Die einfachste Form der objektrelationalen Abbildung ist ein Objekt, dessen
Instanzen Tabellenzeilen entsprechen und dessen Attribute jeweils einer
Tabellenspalte entsprechen. Attribute, die andere Objekte referenzieren, werden
dabei in der Datenbank meistens zusätzlich zur Tabellenspalte durch einen
Fremdschlüssel realisiert.



\section{Vor- und Nachteile von ORM}

Wie fast jede Technik hat natürlich auch ORM Vor- und Nachteile.

Der sicherlich größte Vorteil ist, dass sich der Programmierer viel Arbeit
spart. Er muss weder SQL-Abfragen noch Kapselfunktionen für häufig genutzte
Abfragen schreiben sondern kann einfach mit den meistens sowieso schon
vorhandenen Klassen/Objekten arbeiten. Allerdings hat diese Einfachheit auch
ihren Preis: Während einfache Abfragen von ORM-Systemen ohne Probleme in
performantes SQL umgesetzt werden, welches sich größtenteils nicht von dem
unterscheidet, das ein fähiger Entwickler ansonsten manuell schreiben würde,
so ist es gut möglich dass komplexe Abfragen, die sich über mehrere Tabellen
erstrecken, Aggregationsoperationen enthalten, usw., zwar korrekt ausgeführt
werden, allerdings nicht sehr performant sind. Allerdings bieten ORM-Systeme
für solche Fälle normalerweise eine Möglichkeit an, sowohl Metadaten anzugeben -
entweder beim Erstellen der Abfrage oder schon in der Definition der
Datenbankstruktur - welche die SQL-Generierung (positiv) beeinflussen können als
auch SQL-Abfragen händisch zu schreiben oder zumindest mit einer speziellen
Syntax genau anzugeben wie die SQL-Abfrage generiert werden soll, sodass man
das objektrelationale Mapping hat und trotzdem performantes/hochoptimiertes SQL
nutzen kann.

Ein gutes Beispiel dazu ist \emph{lazy-loading} von Relationen: Angenommen man
hat eine Tabelle, die Autoren enthält, und eine weitere Tabelle, die Bücher
enthält. Diese beiden Tabellen seien über eine Fremdschlüsselbeziehung
miteinander verbunden, sodass ein Buch immer einem Autor zugeordnet ist.
Nun lässt man das ORM-System einen bestimmten Autoren aus der Datenbank
laden. Wenn \emph{lazy-loading} benutzt wird, hat der Autor zwar ein Feld
\emph{Bücher}, allerdings werden seine Bücher erst in dem Moment aus der
Datenbank ausgelesen, wo man lesend auf das Feld zugreift. Dies hat den Vorteil,
dass - wenn man nur den Autor benötigt - keine unnötigen Daten ausgelesen
werden. Wenn man nun aber sowohl den Autor als auch seine Bücher benötigt werden
insgesamt zwei SQL-Abfragen ausgeführt: Zuerst wird auf die Autorentabelle
zugegriffen, dann auf die Büchertabelle. Dasselbe ließe sich allerdings auch in
einer einzelnen Abfrage mittels eines JOINs realisieren - da das ORM-System
allerdings nicht wissen kann, dass auch die Daten aus der referenzierten Tabelle
benötigt, muss man ihm explizit mitteilen, dass es Relationen nicht \emph{lazy}
sondern \emph{eager} laden soll.

Ein anderer klarer Vorteil eines ORM-Systems ist dass man sich um SQL Injection
\footnote{Ausnutzung einer Sicherheitslücke die entsteht, wenn ein regulärer
Benutzer SQL-Abfragen so manipulieren kann, dass sie nicht die gewünsche
Funktion haben sondern eine, die dem Benutzer ungewollte Vorteile verschafft
oder Daten zerstört - siehe auch \autoref{img.exploits-of-a-mom.png}.} keine
Gedanken machen muss: Wenn das ORM-System Abfragen
generiert benutzt es entweder eine Datenbank-API bei der Benutzerdaten und
SQL-Abfrage separat übertragen werden oder aber es escaped Sonderzeichen in den
Benutzerdaten, die die SQL-Abfrage verändern würden.

\bild{exploits-of-a-mom.png}{\textwidth}{SQL Injection}{SQL Injection (von
www.xkcd.com)}

Ein weiterer Punkt, der sowohl Vor- als auch Nachteil ist, ist die zusätzliche
Ebene zwischen dem eigentlichen Programm und der Datenbank (bzw. dem
Datenbank-Client). Der klare Vorteil davon ist, dass die Datenbank (sofern man
nur generiertes SQL nutzt und keine datenbankspezifischen Features nutzt) ohne
viel Aufwand durch eine andere ersetzt werden kann. Der Nachteil ist, dass durch
die zusätzliche Schicht ein Overhead ensteht wo man von Fall zu Fall entscheiden
muss ob er vertretbar ist oder nicht.



\section{Konventioneller Datenbankzugriff in Python}

Python selbst hat bereits eine
Datenbank-API\footnote{\href{http://www.python.org/dev/peps/pep-0249/}{Python
Database API Specification 2.0}} um eine vom Datenbanksystem unabhängige
Zugriffsmöglichkeit auf Datenbanken zu bieten.
Allerdings dient diese API nur dem Zugriff auf die Datenbank sodass man selbst
dafür verantwortlich ist, mit der genutzten Datenbank kompatibles SQL zu
schreiben.

\lstinputlisting[language=Python,label=lst:dbapi,caption=Datenbankzugriff via
DB-API]{code/python-dbapi.py}

\autoref{lst:dbapi} zeigt Python-Code, der über die DB-API sowohl aus
einer SQLite-Datenbank als auch aus einer PostgreSQL-Datenbank Daten ausliest.
Man sieht schnell, dass zwar die API selbst für beide Datenbanksysteme identisch
ist, die Funktionsparameter jedoch nicht: Während \emph{psycopg} \textbf{\%s}
als Platzhalter erwartet, benötigt \emph{sqlite3} \textbf{?} als Platzhalter.

Wenn man jedoch sowieso nur ein einziges Datenbanksystem benutzt und auch nicht
viele Datenbankoperationen benötigt ist die DB-API jedoch oft ausreichend:
Simple SQL-Abfragen sind schnell geschrieben und jeder, der den Code liest,
sieht sofort was er macht. Darüberhinaus muss man sich um SQL-Injection keine
Gedanken machen, da Query und Parameter separat angegeben werden. Allerdings
besteht das Risiko dass gerade Programmierer, die die DB-API zum ersten Mal
benutzen die in \autoref{lst:dbapi-injection} gezeigten Fehler begehen und
dadurch SQL-Injection ermöglichen.

\lstinputlisting[language=Python,label=lst:dbapi-injection,caption=Fehlerhafte
Parameter\"ubergabe an die DB-API]{code/python-dbapi-injection.py}

\chapter{SQLAlchemy}

\section{Allgemeines}

SQLAlchemy\footnote{\href{http://www.sqlalchemy.org}{http://www.sqlalchemy.org}}
ist nicht nur ein ORM-System sondern eine Datenbankabstraktionsebene die auch
ORM kann - man kann es auch problemlos nutzen ohne auch nur ein einziges Objekt
zu mappen.

Das Framework besteht aus drei großen Modulen: \textbf{Dialects}, \textbf{Core}
und \textbf{ORM}.

Das \textbf{Dialects}-Modul ist dafür zuständig, auf die verschiedenen
Datenbanksysteme zuzugreifen - eine vollständige Liste der Datenbanksysteme und
der verschiedenen APIs (für MSSQL\footnote{Microsoft SQL Server} gibt es
beispielsweise 5 verschiedene APIs) findet man in der
\href{http://www.sqlalchemy.org/docs/core/engines.html#supported-dbapis}
{SQLAlchemy-Dokumentation}. Er abstrahiert also den Zugriff auf die
verschiedenen Datenbanken vollständig, sodass anders als bei Pythons BB-API
abgesehen von den Verbindungsdaten keine unterschiedlichen Funktionsparameter
notwendig sind. Eine weitere Aufgabe des \emph{Dialects}-Moduls ist die
Abstraktion der Spaltentypen - die meisten Datenbanksystemen haben zusätzliche
Typen die nicht standardisiert sind; so hat PostgreSQL beispielsweise den
\emph{INET}-Datentyp um IP-Adressen zu speichern - dieser wird durch
\texttt{postgresql.INET} aus \texttt{sqlalchemy.dialects} repräsentiert.

Das \textbf{Core}-Modul enthält alle nicht-ORM-bezogenen Funktionen von
SQLAlchemy. Dazu gehört insbesondere die \emph{SQL Expression Language}, die dem
Entwickler ermöglicht, SQL zu erzeugen ohne selbst SQL zu schreiben, indem sie
die Bestandteile von SQL durch datenbanksystemunabhängige Python-Konstrukte
repräsentiert. Dabei ist es sowohl möglich, nur die Python-Konstrukte zu
benutzen oder einen String mit einer SQL-Abfrage zu übergeben, als auch beides
zu kombinieren. Die beiden letzteren Methoden sind allerdings u.U. nicht mehr
mit allen Datenbanksystemen kompatibel da man nicht ausschließlich die
Abstraktion benutzt sondern (teilweise) selbst SQL-Abfragen schreibt. Ein
weiterer wichtiger Teil vom \emph{Core} ist der Verbindungsaufbau anhand eines
Connectionstrings, der alle benötigten Daten einschließlich Datenbanksystem und
-API enthält. Ebenfalls vom \emph{Core} zur Verfügung gestellt wird der
Connection-Pool, der bereits aufgebaute Datenbankverbindungen enthält sodass
eine Applikation schnell an eine Verbindung kommt ohne erst eine neue Verbindung
aufbauen zu müssen, was je nach Datenbanksystem ein relativ "teurer" Prozess
ist.

Das dritte und wichtigste Modul ist das \textbf{ORM}-Modul. Im Gegensatz zum
Lowlevel-Zugriff den die \emph{SQL Expression Language} bietet ist das ORM auf
einer hohen Ebene angesiedelt und bietet eine starke Abstraktion. Dazu bietet es
verschiedene Möglichkeiten, Datenbanktabellen und Objekte miteinander zu
verbinden - zum einen den \emph{mapper} umd zum anderen die \emph{declarative
syntax}. Ebenfalls im \emph{ORM}-Modul enthalten ist die
\emph{\texttt{relationship()} API}, mit der man Relationen zwischen mehreren
Tabellen mappen kann.


\section{Datenbankdefinition}

Wie bereits erwähnt gibt es zwei Möglichkeiten, die Datenbankstruktur und das
dazugehörige Mapping zu definieren. Als Beispiel für die beiden Möglichkeiten
wird jeweils die folgende Beispieldatenbank dienen:

\underline{Tabelle \textbf{authors}:}\\
\textbf{id} \texttt{INTEGER NOT NULL}\\
name \texttt{VARCHAR NOT NULL}

\underline{Tabelle \textbf{books}:}\\
\textbf{id} \texttt{INTEGER NOT NULL}\\
\textit{author\_id} \texttt{INTEGER NOT NULL}\\
title \texttt{VARCHAR NOT NULL}

\underline{Tabelle \textbf{tags}:}\\
\textbf{id} \texttt{INTEGER NOT NULL}\\
tag \texttt{VARCHAR UNIQUE NOT NULL}

\underline{Tabelle \textbf{tags2books}:}\\
\textbf{\textit{tag\_id}} \texttt{INTEGER NOT NULL}\\
\textbf{\textit{book\_id}} \texttt{INTEGER NOT NULL}

\underline{Fremdschlüssel-Beziehungen:}\\
books.author\_id $\rightarrow$ authors.id\\
tags2books.tag\_id -> tags.id\\
tags2books.author\_id -> authors.id

\subsection{Standardsyntax}
\subsection{Deklarative Syntax}

\section{Datenbankzugriff}
\subsection{Zugriff via ORM}
\subsection{Lowlevel-Zugriff}

\section{Deklarationsfreies ORM mit SqlSoup}


\chapter{SQLAlchemy - praktische Nutzung}

Die derzeit noch in Entwicklung befindliche neue Version der Websites des
\href{http://www.gamesurge.net}{GameSurge IRC-Netzes} verwendet eine angepasste
Version des \href{https://github.com/gmr/tinman}{Tinman-Frameworks}, welches auf
dem \href{http://www.tornadoweb.org/}{Tornado-Framework/Webserver} von Facebook
basiert. Die wichtigste Anpassung bei Tinman betrifft den SQLAlchemy-Code des
Datenbankmoduls, der standardmäßig immer \texttt{autocommit} benutzt, was wie in
\autoref{sqlalchemy-orm} erwähnt problematisch sein kann. In der angepassten
Version kann \texttt{autocommit} per Configdatei deaktiviert werden, was auch
notwendig ist, damit die Website korrekt funktioniert.

Das erste Beispiel - die dazugehörige Tabelle wird in
\autoref{lst:gs-model-trustrequest} definiert - ist aus einem Interface, über
das Benutzer gewisse Privilegien, sog. \emph{Trusts}, für ihre IP-Adresse
beantragen können. Da die Website PostgreSQL als Datenbankbackend nutzt und sich
dies mit größter Wahrscheinlichkeit nicht ändern wird, wird die IP-Adresse in
einem Datenbankfeld des Typs \texttt{postgresql.INET} gespeichert. Am Namen des
Datentypen sieht man bereits, dass es sich hierbei um einen
datenbankspezifischen Typen handelt, der nicht mit anderen Datenbanksystemen
funktionieren wird. Allerdings hat er den Vorteil, dass er sowohl IPv4- als auch
IPv6\footnote{\emph{Internet Protocol Version 6}, dringend benötigter Nachfolger
des aktuellen IPv4 mit einem weitaus größeren Adressraum von $2^{128}$ statt
$2^{32}$}-Adressen effizient speichern kann und IP-spezifische Operationen wie
eine Suche nach IPs innerhalb eines gewissen Subnetzes unterstützt.
Der Parameter \texttt{server\_default} der Spalte \texttt{request\_time} gibt
an, dass beim Erstellen der Tabelle ein serverseitiger Defaultwert angegeben
wird, der durch die SQL-Funktion \texttt{current\_timestamp()} erzeugut wird.
Das zum Erstellen der Tabelle generierte SQL enthält also \texttt{DEFAULT
current\_timestamp()}. Alternativ zum \texttt{server\_default}-Parameter gäbe es
auch \texttt{default}. Diese Option würde der Spalte in der Datenbank keinen
Defaultwert zuweisen sondern ihn explizit im \texttt{INSERT}-Statement setzen.

\lstinputlisting[language=Python,label=lst:gs-model-trustrequest,
caption=TrustRequest-Tabellendefinition]{code/gs-model-trustrequest.py}

\autoref{lst:gs-trust-request} ist der Code hinter dem für Benutzer zugänglichen
Formular. Um zu verhindern, dass ein \emph{Trust} erneut beantragt wird, solange
der erste Antrag noch nicht bearbeitet wurde, wird die Anzahl der offenen
Anträge für die angegebene IP aus der Datenbank ausgelesen und sofern sie nicht
\texttt{0} ist mit einer Fehlermeldung abgebrochen. Danach wird das neue
\texttt{TrustRequest}-Objekt erstellt und mit Daten gefüllt. Da zusätzlich noch
Daten zu der IP abgefragt werden sollen, die u.U. längere Netzwerkanfragen
benötigen, wird das Objekt noch nicht persistiert sondern zuerst an asynchrone
Funktionen übergeben, die diese Daten abfragen. Nachdem alle Daten abgefragt
und dem Objekt hinzugefügt wurden, wird das Objekt in der Methode
\texttt{post\_resolved()} der Session hinzugefügt und die Datenbanktransaktion
committed.

\lstinputlisting[language=Python,label=lst:gs-trust-request,
caption=Erstellen eines neuen TrustRequest-Objekts]{code/gs-trust-request.py}

Beim Bearbeiten der Anträge ist es oftmals von Interesse, ob der Benutzer
schon andere \emph{Trusts} beantragt hat und ob für die angegebene IP-Adresse
bereits \emph{Trusts} beantragt wurden. Diese Informationen werden vom Code in
\autoref{lst:gs-trust-list} abgerufen. In der dort verwendeten Query sieht man
gut die Vermischung von ORM und klassischen SQL-Konstrukten für komplexe
Abfragen: Die \texttt{TrustRequest}-Objekte selbst werden über die normalen
ORM-Funktinen aus der Datenbank abgerufen. Allerdings wird die Query um
\texttt{ip\_reqs.as\_scalar()} und \texttt{user\_reqs.as\_scalar()} erweitert.
Die Methode \texttt{as\_scalar()} gibt ein \texttt{Query}-Objekt in einer
Subquery-kompatiblen Form zurück. Ebenfalls erwähnenswert ist das Erstellen der
Subqueries. Da es notwendig ist, in der \texttt{WHERE}-Bedingung der Subqueries
eine Spalte aus derselben Tabelle der Haupt-Query zu referenzieren, muss die
Tabelle der Subquery über einen Alias referenziert werden. Dies geschieht mit
der \texttt{aliased()}-Funktion.
Beim Iterieren über das Resultset bekommt man für die einzelnen Zeilen nun nicht
mehr einfach das Mapping-Objekt sondern ein Tupel, welches alle im
\texttt{query()}-Aufruf angegebenen Elemente enthält, d.h. das Mapping-Objekt
für den \texttt{TrustRequest} und von \texttt{ip\_reqs} und \texttt{user\_reqs}
jeweils einen Integer.

\lstinputlisting[language=Python,label=lst:gs-trust-list,
caption=Abrufen von TrustRequest-Objekten und zus\a"tzlichen
Informationen]{code/gs-trust-list.py}

Die Website soll ebenfalls die Möglichkeit bieten, News zu posten und
nachträglich in andere Sprachen zu übersetzen. Dabei sollen die
englischsprachigen News und die Übersetzungen in verschiedenen Tabellen
gespeichert werden. Um anhand einem \texttt{News}-Objekts leicht an eine
bestimmte Übersetzung zu kommen ist es wünschenswert, bei der Relation von
\texttt{News} nach \texttt{NewsTranslation} keine Liste mit numerischen Indizes
sondern ein \texttt{dict} mit dem Sprachkürzel als Index zu haben. Dies wird in
\autoref{lst:gs-model-news} mithilfe des \texttt{collection\_class}-Parameters
von \texttt{relationship()} realisiert. \\
\texttt{attribute\_mapped\_collection('language')} gibt an, dass das Feld
\texttt{language} als Index benutzt werden soll. \\
Zusätzlich wurde mit
\texttt{cascade='all, delete, delete-orphan'} festgelegt, dass "verwaiste"
Übersetzungen, d.h. Übersetzungen deren Original-News gelöscht wurde, ebenfalls
gelöscht werden. Der Fremdschlüssel übernimmt dies dank
\texttt{ondelete='CASCADE'} zwar bereits, allerdings würde SQLAlchemy ohne
die Kaskadierung versuchen, die \texttt{newsid} der \texttt{NewsTranslation} auf
\texttt{NULL} zu setzen, was an \texttt{nullable=False} scheitern würde und auch
nicht gewollt ist.

\lstinputlisting[language=Python,label=lst:gs-model-news,
caption=News-Tabellendefinition]{code/gs-model-news.py}

Im News-Template (\autoref{lst:gs-news-tpl}) wird bei der Ausgabe dann einfach
geprüft, ob es für die aktuelle Sprache eine Übersetzung gibt und danach
entweder die Übersetzung oder das englische Original ausgegeben.

\lstinputlisting[language=Python,label=lst:gs-news-tpl,caption=News-Template]
{code/gs-news-tpl.html}

\chapter{Ausblick}

\section{Verbreitung}

\section{Unterschiede zu anderen ORMs}

\section{Fazit}


\clearpage
\end{document}
