\chapter{Grundlagen}

\section{Was ist ORM?}

Unter objektrelationaler Abbildung
(engl. \textbf{o}bject-\textbf{r}elational \textbf{m}apping)
versteht man eine Technik, mit der man in einer objektorientierten
Programmiersprache Objekte in einer relationalen Datenbank ablegen kann und
später wieder darauf zugreifen kann. Dazu ist es normalerweise nicht
notwendig, Details über die tatsächliche Datenbankstruktur zu kennen oder gar
selbst SQL\footnote{\emph{Structured Query Language}, eine standardisierte
Datenbanksprache}-Abfragen zu schreiben.

Die einfachste Form der objektrelationalen Abbildung ist ein Objekt, dessen
Instanzen Tabellenzeilen entsprechen und dessen Attribute jeweils einer
Tabellenspalte entsprechen. Attribute, die andere Objekte referenzieren, werden
dabei in der Datenbank meistens zusätzlich zur Tabellenspalte durch einen
Fremdschlüssel realisiert.



\section{Vor- und Nachteile von ORM}

Wie fast jede Technik hat natürlich auch ORM Vor- und Nachteile.

Der sicherlich größte Vorteil ist, dass sich der Programmierer viel Arbeit
spart. Er muss weder SQL-Abfragen noch Kapselfunktionen für häufig genutzte
Abfragen schreiben sondern kann einfach mit den meistens sowieso schon
vorhandenen Klassen/Objekten arbeiten. Allerdings hat diese Einfachheit auch
ihren Preis: Während einfache Abfragen von ORM-Systemen ohne Probleme in
performantes SQL umgesetzt werden, welches sich größtenteils nicht von dem
unterscheidet, das ein fähiger Entwickler ansonsten manuell schreiben würde,
so ist es gut möglich dass komplexe Abfragen, die sich über mehrere Tabellen
erstrecken, Aggregationsoperationen enthalten, usw., zwar korrekt ausgeführt
werden, allerdings nicht sehr performant sind. Allerdings bieten ORM-Systeme
für solche Fälle normalerweise eine Möglichkeit an, sowohl Metadaten anzugeben -
entweder beim Erstellen der Abfrage oder schon in der Definition der
Datenbankstruktur - welche die SQL-Generierung (positiv) beeinflussen können als
auch SQL-Abfragen händisch zu schreiben oder zumindest mit einer speziellen
Syntax genau anzugeben wie die SQL-Abfrage generiert werden soll, sodass man
das objektrelationale Mapping hat und trotzdem performantes/hochoptimiertes SQL
nutzen kann.

Ein gutes Beispiel dazu ist \emph{lazy-loading} von Relationen: Angenommen man
hat eine Tabelle, die Autoren enthält, und eine weitere Tabelle, die Bücher
enthält. Diese beiden Tabellen seien über eine Fremdschlüsselbeziehung
miteinander verbunden, sodass ein Buch immer einem Autor zugeordnet ist.
Nun lässt man das ORM-System einen bestimmten Autoren aus der Datenbank
laden. Wenn \emph{lazy-loading} benutzt wird, hat der Autor zwar ein Feld
\emph{Bücher}, allerdings werden seine Bücher erst in dem Moment aus der
Datenbank ausgelesen, wo man lesend auf das Feld zugreift. Dies hat den Vorteil,
dass - wenn man nur den Autor benötigt - keine unnötigen Daten ausgelesen
werden. Wenn man nun aber sowohl den Autor als auch seine Bücher benötigt werden
insgesamt zwei SQL-Abfragen ausgeführt: Zuerst wird auf die Autorentabelle
zugegriffen, dann auf die Büchertabelle. Dasselbe ließe sich allerdings auch in
einer einzelnen Abfrage mittels eines JOINs realisieren - da das ORM-System
allerdings nicht wissen kann, dass auch die Daten aus der referenzierten Tabelle
benötigt, muss man ihm explizit mitteilen, dass es Relationen nicht \emph{lazy}
sondern \emph{eager} laden soll.

Ein anderer klarer Vorteil eines ORM-Systems ist dass man sich um SQL Injection
\footnote{Ausnutzung einer Sicherheitslücke die entsteht, wenn ein regulärer
Benutzer SQL-Abfragen so manipulieren kann, dass sie nicht die gewünsche
Funktion haben sondern eine, die dem Benutzer ungewollte Vorteile verschafft
oder Daten zerstört - siehe auch \autoref{img.exploits-of-a-mom.png}.} keine
Gedanken machen muss: Wenn das ORM-System Abfragen
generiert benutzt es entweder eine Datenbank-API bei der Benutzerdaten und
SQL-Abfrage separat übertragen werden oder aber es escaped Sonderzeichen in den
Benutzerdaten, die die SQL-Abfrage verändern würden.

\bild{exploits-of-a-mom.png}{\textwidth}{SQL Injection (von
\href{http://xkcd.com/327/}{xkcd.com})}{SQL Injection}

Ein weiterer Punkt, der sowohl Vor- als auch Nachteil ist, ist die zusätzliche
Ebene zwischen dem eigentlichen Programm und der Datenbank (bzw. dem
Datenbank-Client). Der klare Vorteil davon ist, dass die Datenbank (sofern man
nur generiertes SQL nutzt und keine datenbankspezifischen Features nutzt) ohne
viel Aufwand durch eine andere ersetzt werden kann. Der Nachteil ist, dass durch
die zusätzliche Schicht ein Overhead ensteht wo man von Fall zu Fall entscheiden
muss ob er vertretbar ist oder nicht.



\section{Konventioneller Datenbankzugriff in Python}

Python selbst hat bereits eine
Datenbank-API\footnote{\href{http://www.python.org/dev/peps/pep-0249/}{Python
Database API Specification 2.0}} um eine vom Datenbanksystem unabhängige
Zugriffsmöglichkeit auf Datenbanken zu bieten.
Allerdings dient diese API nur dem Zugriff auf die Datenbank sodass man selbst
dafür verantwortlich ist, mit der genutzten Datenbank kompatibles SQL zu
schreiben.

\autoref{lst:dbapi} zeigt Python-Code, der über die DB-API sowohl aus
einer SQLite-Datenbank als auch aus einer PostgreSQL-Datenbank Daten ausliest.
Man sieht schnell, dass zwar die API selbst für beide Datenbanksysteme identisch
ist, die Funktionsparameter jedoch nicht: Während \emph{psycopg} \textbf{\%s}
als Platzhalter erwartet, benötigt \emph{sqlite3} \textbf{?} als Platzhalter.

\lstinputlisting[language=Python,label=lst:dbapi,caption=Datenbankzugriff via
DB-API]{code/python-dbapi.py}

Wenn man allerdings sowieso nur ein einziges Datenbanksystem benutzt und auch
nicht viele Datenbankoperationen benötigt ist die DB-API jedoch oft ausreichend:
Simple SQL-Abfragen sind schnell geschrieben und jeder, der den Code liest,
sieht sofort was er macht. Darüberhinaus muss man sich um SQL-Injection keine
Gedanken machen, da Query und Parameter separat angegeben werden. Allerdings
besteht das Risiko dass gerade Programmierer, die die DB-API zum ersten Mal
benutzen die in \autoref{lst:dbapi-injection} gezeigten Fehler begehen und
dadurch SQL-Injection ermöglichen.

\lstinputlisting[language=Python,label=lst:dbapi-injection,caption=Fehlerhafte
Parameter\"ubergabe an die DB-API]{code/python-dbapi-injection.py}
