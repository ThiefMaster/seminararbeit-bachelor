\chapter{SQLAlchemy}

\section{Allgemeines}

SQLAlchemy\footnote{\href{http://www.sqlalchemy.org}{http://www.sqlalchemy.org}}
ist nicht nur ein ORM-System sondern eine Datenbankabstraktionsebene die auch
ORM kann - man kann es auch problemlos nutzen ohne auch nur ein einziges Objekt
zu mappen.

Das Framework besteht aus drei großen Modulen: \textbf{Dialects}, \textbf{Core}
und \textbf{ORM}.

Das \textbf{Dialects}-Modul ist dafür zuständig, auf die verschiedenen
Datenbanksysteme zuzugreifen - eine vollständige Liste der Datenbanksysteme und
der verschiedenen APIs (für MSSQL\footnote{Microsoft SQL Server} gibt es
beispielsweise 5 verschiedene APIs) findet man in der
\href{http://www.sqlalchemy.org/docs/core/engines.html#supported-dbapis}
{SQLAlchemy-Dokumentation}. Er abstrahiert also den Zugriff auf die
verschiedenen Datenbanken vollständig, sodass anders als bei Pythons BB-API
abgesehen von den Verbindungsdaten keine unterschiedlichen Funktionsparameter
notwendig sind. Eine weitere Aufgabe des \emph{Dialects}-Moduls ist die
Abstraktion der Spaltentypen - die meisten Datenbanksystemen haben zusätzliche
Typen die nicht standardisiert sind; so hat PostgreSQL beispielsweise den
\emph{INET}-Datentyp um IP-Adressen zu speichern - dieser wird durch
\texttt{postgresql.INET} aus \texttt{sqlalchemy.dialects} repräsentiert.

Das \textbf{Core}-Modul enthält alle nicht-ORM-bezogenen Funktionen von
SQLAlchemy. Dazu gehört insbesondere die \emph{SQL Expression Language}, die dem
Entwickler ermöglicht, SQL zu erzeugen ohne selbst SQL zu schreiben, indem sie
die Bestandteile von SQL durch datenbanksystemunabhängige Python-Konstrukte
repräsentiert. Dabei ist es sowohl möglich, nur die Python-Konstrukte zu
benutzen oder einen String mit einer SQL-Abfrage zu übergeben, als auch beides
zu kombinieren. Die beiden letzteren Methoden sind allerdings u.U. nicht mehr
mit allen Datenbanksystemen kompatibel da man nicht ausschließlich die
Abstraktion benutzt sondern (teilweise) selbst SQL-Abfragen schreibt. Ein
weiterer wichtiger Teil vom \emph{Core} ist der Verbindungsaufbau anhand eines
Connectionstrings, der alle benötigten Daten einschließlich Datenbanksystem und
-API enthält. Ebenfalls vom \emph{Core} zur Verfügung gestellt wird der
Connection-Pool, der bereits aufgebaute Datenbankverbindungen enthält sodass
eine Applikation schnell an eine Verbindung kommt ohne erst eine neue Verbindung
aufbauen zu müssen, was je nach Datenbanksystem ein relativ "teurer" Prozess
ist.

Das dritte und wichtigste Modul ist das \textbf{ORM}-Modul. Im Gegensatz zum
Lowlevel-Zugriff den die \emph{SQL Expression Language} bietet ist das ORM auf
einer hohen Ebene angesiedelt und bietet eine starke Abstraktion. Dazu bietet es
verschiedene Möglichkeiten, Datenbanktabellen und Objekte miteinander zu
verbinden - zum einen den \emph{mapper} umd zum anderen die \emph{declarative
syntax}. Ebenfalls im \emph{ORM}-Modul enthalten ist die
\emph{\texttt{relationship()} API}, mit der man Relationen zwischen mehreren
Tabellen mappen kann.


\section{Datenbankdefinition}

Wie bereits erwähnt gibt es zwei Möglichkeiten, die Datenbankstruktur und das
dazugehörige Mapping zu definieren. Als Beispiel für die beiden Möglichkeiten
wird jeweils die folgende Beispieldatenbank dienen:

\underline{Tabelle \textbf{authors}:}\\
\textbf{id} \texttt{INTEGER NOT NULL}\\
name \texttt{VARCHAR NOT NULL}

\underline{Tabelle \textbf{books}:}\\
\textbf{id} \texttt{INTEGER NOT NULL}\\
\textit{author\_id} \texttt{INTEGER NOT NULL}\\
title \texttt{VARCHAR NOT NULL}

\underline{Tabelle \textbf{tags}:}\\
\textbf{id} \texttt{INTEGER NOT NULL}\\
tag \texttt{VARCHAR UNIQUE NOT NULL}

\underline{Tabelle \textbf{tags2books}:}\\
\textbf{\textit{tag\_id}} \texttt{INTEGER NOT NULL}\\
\textbf{\textit{book\_id}} \texttt{INTEGER NOT NULL}

\underline{Fremdschlüssel-Beziehungen:}\\
books.author\_id $\rightarrow$ authors.id\\
tags2books.tag\_id -> tags.id\\
tags2books.author\_id -> authors.id

Bei den Primärschlüsseln in \emph{authors}, \emph{books} und \emph{tags} sollen
darüberhinaus automatisch generierte IDs benutzt werden.

\subsection{mapper()-Syntax}

Nutzt man die klassische \emph{mapper()-Syntax} so definiert man Tabellen,
Klassen und Mappings/Relationen nacheinander.
\autoref{lst:dbdef-mapper-tables} zeigt die Definition der Tabellen und
datenbankseitigen Relationen (Fremdschlüssel, Indizes). Im zuvor erzeugten
\texttt{metadata}-Objekt legt SQLAlchemy die Metadaten zu den Tabellen und
Relationen ab. Desweiteren kann man über dieses Objekt die Tabellen in der
Datenbank erstellen (\texttt{metadata.create\_all()}) und auch löschen lassen.
Da die \emph{id}-Spalten als \texttt{Integer} und \texttt{primary\_key}
definiert sind generiert SQLAlchemy automatisch eine automatisch
inkrementierende Spalte - d.h. in PostgreSQL eine Spalte vom Typ \texttt{serial}
und in MySQL eine \texttt{INT}-Spalte mit dem \texttt{AUTO\_INCREMENT}-Attribut.
Damit die Spalten nicht \texttt{NULL} sein dürfen wurde bei allen
nicht-Primärschlüssel-Spalten das \texttt{nullable}-Attribut auf \texttt{False}
gesetzt. Bei den Primärschlüsseln ist dies nicht notwendig, da ein
Primärschlüssel \textit{per definitionem} nicht \texttt{NULL} sein kann.
Ebenfalls in der Tabellendefinition enthalten sind Fremdschlüssel, die mit dem
\texttt{ForeignKey()}-Konstrukt definiert werden, und reguläre Indizes, die mit
\texttt{index=True} definiert werden.

\lstinputlisting[language=Python,label=lst:dbdef-mapper-tables,lastline=30,
caption=Definition der Tabellen]{code/sqlalchemy-def-mapper.py}

\todotext{Klassendefinition - \autoref{lst:dbdef-mapper-classes}}

\lstinputlisting[language=Python,label=lst:dbdef-mapper-classes,firstline=32,
lastline=43,caption=Definition der Klassen]{code/sqlalchemy-def-mapper.py}

\todotext{Mapping - \autoref{lst:dbdef-mapper-mapping}}

\lstinputlisting[language=Python,label=lst:dbdef-mapper-mapping,firstline=45,
lastline=52,caption=Mapping der Tabellen auf die
Klassen]{code/sqlalchemy-def-mapper.py}


\subsection{Deklarative Syntax}

\section{Datenbankzugriff}
\subsection{Zugriff via ORM}
\subsection{Lowlevel-Zugriff}

\section{Deklarationsfreies ORM mit SqlSoup}

